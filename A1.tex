\documentclass[11pt, answers]{exam}
\renewcommand{\baselinestretch}{1.05}
\usepackage{amsmath,amsthm,verbatim,amssymb,amsfonts,amscd, graphicx}
\usepackage{graphics}

\usepackage{afterpage}
\usepackage{caption}

\usepackage{fancybox}

\usepackage{clrscode3e}

\topmargin0.0cm
\headheight0.0cm
\headsep0.0cm
\oddsidemargin0.0cm
\textheight23.0cm
\textwidth16.5cm
\footskip1.0cm
\theoremstyle{plain}
\newtheorem{theorem}{Theorem}
\newtheorem{corollary}{Corollary}
\newtheorem{lemma}{Lemma}
\newtheorem{proposition}{Proposition}
\newtheorem*{surfacecor}{Corollary 1}
\newtheorem{conjecture}{Conjecture}  
\theoremstyle{definition}
\newtheorem{definition}{Definition}

\title{CSC336: Assignment 1}
\author{Junjie Cheng, 1002770539}
\date{October 12 2017}

\begin{document}

\maketitle

\begin{questions}
\question %Q1
\begin{parts}
\part
Absolute error$=A-T=2.72 - 2.71828182845905 = 1.72* 10^{-3}$\\
Relative error$=\frac{A-T}{T} =6.32 * 10^{-4} $

\part
Absolute error:$=A-T=2.718 - 2.71828182845905 = -2.82 * 10^{-4}$\\
Relative error:$\frac{A-T}{T} = -1.04 * 10^{-4}$

\part
Absolute error:$=A-T=2.71828183 - 2.71828182845905=1.54*10^{-9}$\\
Relative error:$=\frac{A-T}{T} = 5.67*10^{-10}$

\end{parts}
\question %Q2
\begin{parts}
\part
$4.21*10^0 + 5.47*10^{-2} = 4.2647*10^0$. It will be rounded to $4.26*10^0$

\part
$6.52*10^1 - 7.27*10^10^{-1} =  6.4473*10^1$. It will be rounded to $6.45 * 10^1$

\part
$5.61*10^1 + 6.67*10^{-4} = 5.6100667 * 10^1$. It will be rounded to $5.61*10^1$

\part
$4.52*10^4 - 3.82*10^6 = -3.7748 * 10^6$. It will be rounded to $-3.77*10^6$

\part
$7.51*10^12 - 5.25*10^5$ will be rounded to $7.51*10^{12}$

\part
$3.82*10^1 + 8.42*10^2 = 8.803 * 10^2$. It will be rounded to $8.80*10^2$

\part
$4.47 * 10^{10} * 5.81 * 10^{−15} = 2.59707*10^{-4}$. It will be rounded to $2.60 * 10^{-4}$

\part
$2.41 * 10^{−10} * 4.81 * 10^{−12} = 1.15921*10^{-21} = 0.115921*10^{-20}$. It will be rounded to $0.12*10^{-20}$, a subnormal floating point.

\part
$−6.37 * 10^{−10} * 5.28 * 10*{−15} = -3.36336 * 10 ^{-24}$. It will be rounded to 0 because of underflow.

\part
$−6.27 * 10^{10} / (2.72 * 10^{−15}) = -2.305*10^{25}$. It will be rounded to -Inf because of overflow.
\end{parts}
\question %Q3
\begin{parts}
\part
Let $\Delta x$ denote the change of input $x$ (i.e. $\Delta x = \hat{x} - x$).

For very small $\Delta x$, we can write $f(\hat{x}) - f(x) = \Delta x f'(x)$

$
\hspace*{0.4cm}
Relative Forward Error \\
= \frac{f(\hat{x}) - f(x)}{f(x)}\\
= \frac{f(x+\Delta h) - f(x)}{f(x)}
= \frac{f'(x)(\hat{x} - x)}{f(x)}
= \frac{xf'(\hat{x})}{f(x)} \times \frac{\hat{x}- x}{x}
= \frac{1}{\log_e (x)} \times \frac{\hat{x}- x}{x}
$

When $x$ is close to $1$, the condition number $|\frac{1}{\log_e(x)}|$ approaches Inf. The function is ill-conditioned in a relative sense with respect to small relative changes in the value of the input argument $x$ for $x$ close to $1$.

When $x$ is close to $10$, the condition number $|\frac{1}{\log_e(x)}| \approx \frac{1}{\log_e 10} \approx 0.4343$. The function is well-conditioned in a relative sense with respect to small relative changes in the value of the input argument $x$ for $x$ close to $10$.

\part
See attachments for code and result.

In part (a), we calculated that the condition number can (almost) be expressed as $|\frac{1}{\log_e (x)}|$. In the computational result of part (b), the condition numbers are very close for the same $x$. Also, the condition numbers are very large when $x=1$ and small when $x = 10$, this also agrees with the calculation we have done in part (a).

\end{parts}

\question %Q4
\begin{parts}
\part
$
\hspace*{0.4cm}
Relative Error  \\
= \frac{(\frac{1}{1-x} - \frac{1}{1+x}) - (\frac{1}{(1-x)(1+\sigma_1)}(1+\sigma_2) - \frac{1}{(1+x)(1+\sigma_3)}(1+\sigma_4))(1+\sigma_5)}{\frac{1}{1-x} - \frac{1}{1+x}} \\
= \frac{\frac{2x(1+\sigma_1)(1+\sigma_3) + (1-x)(1+\sigma_1)(1+\sigma_4)(1+\sigma_5) -(1+x)(1+\sigma_3)(1+\sigma_2)(1+\sigma_5)}{(1-x^2)(1+\sigma_1)(1+\sigma_3)}}{\frac{2x}{1-x^2}} \\
= 1+\frac{(1+\sigma_2)(1+\sigma_3)(1+\sigma_5)-(1+\sigma_1)(1+\sigma_4)(1+\sigma_5)}{2} - \frac{(1+\sigma_2)(1+\sigma_3)(1+\sigma_5)+(1+\sigma_1)(1+\sigma_4)(1+\sigma_5)}{2x(1+\sigma_1)(1+\sigma_3)}
$

Note that the last part of the relative error is inverse proportional to $x$. That is, when $x$ is extremely small, the absolute value of relative error can be very large.

\part
We choose $\frac{2x}{(1-x)(1+x)} = \frac{1}{1-x} - \frac{1}{1+x}$. \\

$
\hspace*{0.4cm}
Relative Error \\
= \frac
{\frac{2x}{(1-x)(1+x)} - \frac{(2x)(1+\sigma_1)}{ ((1-x)(1+\sigma_2)(1+x)(1+\sigma_3))(1+\sigma_5)}(1+\sigma_4) }
{\frac{2x}{(1-x)(1+x)}} \\
= 1 - \frac{ \frac{(2x)(1+\sigma_1)}{((1-x)(1+\sigma_2)(1+x)(1+\sigma_3))(1+\sigma_5)}(1+\sigma_4)}{\frac{2x}{(1-x)(1+x)}} \\
= 1 - \frac{(1+\sigma_1)(1+\sigma_4)}{(1+\sigma_2)(1+\sigma_3)(1+\sigma_5)} \\
= \frac{\sigma_2+\sigma_3+\sigma_5 + \sigma_2 \sigma_3+\sigma_2 \sigma_5+ \sigma_3 \sigma_5 + \sigma_2 \sigma_3 \sigma_5 - \sigma_1 - \sigma_4 - \sigma_1 \sigma_4}{1+\sigma_2+\sigma_3+\sigma_5 + \sigma_2 \sigma_3+\sigma_2 \sigma_5+ \sigma_3 \sigma_5 + \sigma_2 \sigma_3 \sigma_5}
$

Note that $|\sigma_i|< \epsilon_{machine}$, so in the worst case, we have 
$$|Relative Error| = \frac {5\epsilon_{machine}+4\epsilon_{machine} ^ 2+\epsilon_{machine}^3}{1-3\epsilon_{machine}-3\epsilon_{machine}^2 - 3\epsilon_{machine} ^3} 
$$
Since $\epsilon_{machine} << 1$, $|Relative Error| <6\epsilon_{machine}$ in the worst case. So, we can conclude that $\frac{2x}{(1-x)(1+x)}$ has a very small relative error for all values of $x$, provided that there is no overflow or underflow.

\end{parts}
\question %Q5
\begin{parts}
\part
See attachments for code and output.

\part
For $x > -18$, the function produce very accurate approximations, (when $x=-18$ the approximation is also kind of accurate as the relative error is around only $0.05$).

For $x < -19$, the relative error gets much larger. Of course rounding error contributes at lot on the poor approximations, the main reason that the function performs well on $x>19$ but badly on $x<-19$ is that, $exp(x)$ gets too small for $x < -19$. When denominator gets too small, the whole fraction denoting the relative number gets large. 

On the other hand, when $|x|$ gets larger, it will require a higher $i$ for $\frac{x^i}{i!}$ to be insignificant so that the accumulated sum stops changing. Much more operations are needed to compute each $\frac{x^i}{i!}$ with a very high $i$, resulting a higher error for $\frac{x^i}{i!}$. This can make the relative error grows even faster for more negative $x$. However, the denominator $exp(x)$ gets too small is the main reason for the poor approximation for $x < -19$.


\part
See attachments for code and output.

\end{parts}
\end{questions}
\end{document}
